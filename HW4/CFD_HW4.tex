
\documentclass[12pt, letterpaper, twoside]{article}
\usepackage[margin=2.50cm]{geometry} % Makes the Margins not stupid
\usepackage{bm} % for making bold equation variables
\usepackage{graphicx} % for placing images
\usepackage{float} % for making figure stay in place
\usepackage{amssymb} % for \lessapprox symbol
\usepackage{amsmath} % because using math and \text command in math mode
\usepackage[numbers]{natbib}
\usepackage{cancel} % for doing a diagonal slash cancel
\usepackage[mathscr]{euscript} % for Fancy curly F
\usepackage{commath} % for \od, \pd
\usepackage{url} % for \path{}

\title{ASEN 5331 - HW4}
\author{James Wright}
% \date{2019-01-21}

%------------ MACROS ----------------
\newcommand{\etot}{e_{tot}}
\renewcommand{\vec}[1]{\bm{#1}}
\newcommand{\ttt}[1]{\texttt{#1}}
\newcommand{\U}{\vec{U}}
\newcommand{\A}{\vec{A}}
\newcommand{\Y}{\vec{Y}}
\newcommand{\F}{\vec{F}}
\newcommand{\G}{\vec{G}}
\newcommand{\M}{\vec{M}}
\newcommand{\Sm}{\vec{S}}
\newcommand{\0}{\vec{0}}
\newcommand{\src}{\vec{\mathscr{F}}}

\begin{document}
\maketitle
\subsection{Meaning of \ttt{n...}}

\begin{tabular} { |c|l|c|}
    \hline
    Term & Definition & Source/Relevant Reference \\
    \hline
    \ttt{nsd} & number of spacial dimensions & \path{common/common.h#343} \\
    \ttt{nflow} & number of flow variables (ie. size of \(\Y\)) & ? \\
    \ttt{nshape} & number of interior element shape functions & \path{common/common.h#444} \\
    \ttt{ngauss} & number of interior element integration points & \path{common/common.h#447} \\
    \ttt{npro} & number of elements processed in a single call of \ttt{e3.f} & Jansen lecture \\
    \ttt{npro} & number of virtual processors for the current block & \path{common/common/h#586} \\
    \ttt{nen} & maximum number of element nodes & \path{common/common.h#341} \\
    \ttt{nQpt} & number of quadrature points per element & \path{common/shp4t.f#14} \\
    \ttt{nshl} & number of shape functions per element & \path{common/genblkPosix.f#70} \\
    \ttt{nshg} & global number of shape functions & \path{common/common.h#354} \\
    \ttt{nenl} & number of element nodes for current block & \path{common/common.h#382} \\
    \ttt{nedof} & total number of degrees of freedom & \path{common/e3.f#35,344} \\

    \hline

\end{tabular}

\section{Essential Boundary Conditions}

\subsection{Setting BC Values}

The essential boundary conditions are set in \path{/compressible/itrbc.f#59-198}. The \ttt{iBC(nshg)} variable contains bit-wise information on what specific boundary conditions are going to be set. \ttt{BC(nshg, ndofBC)} contains the BC data (\(g(x)\) in the notes) for each individual node. \ttt{BC} is the equivalent of \(\hat{\vec{q}}\), where the index of \(\hat{\vec{q}}\) is stored in \ttt{ndofBC}. 
\ttt{iBC} is set in \path{common/genibc.f} and \ttt{BC} is set in \path{common/genbc.f}, which takes \ttt{iBC} as an input. 

Essentially, the code checks \ttt{iBC} for which values of \(\Y\) should be set. This logical check is done via the \ttt{ibits()} function. If a given \(\Y\) chosen, then \(\Y\) is set to the corresponding value in \ttt{BC}.

So \path{common/gendat.f->gendat()} calls \path{common/genibc.f->geniBC()} to create the \ttt{iBC} vector. \path{common/gendat.f} then calls \path{common/genbc.f->genBC()} to create the \ttt{BC} vector which contains the values that should then be substituted into the \ttt{y} array in \path{compressible/itrbc.f}.

\path{proces()} initially calls \path{gendat()}, which then calls \path{itrdrv()->itrbc()}. Note that \ttt{iBC} and \ttt{BC} are a global arrays (length \ttt{nshg}). They are simply used once at the beginning of the solver to set the \(g(x)\) values to \(\Y\) and to set the adjusted weight functions (using the \(\Sm\) matrices, discussed below). This way, the essential BC's don't have to be read and used every time step.


\subsection{Applying \(\vec{S}\) Matrices}
The application of the \(\Sm\) matrices applied in two different locations: \path{compressible/b3res.f} and \path{compressible/b3lhs.f}, which apply \(\Sm\) to the residual (\ttt{res}, RHS) and mass matrix (\ttt{EGMass}, LHS) respectively. 

For the residual, the values of \ttt{res} are replaced based on the logical output from the same \ttt{ibits()} operation on \ttt{iBC} as before, similar to when \ttt{BC} values were set in \path{common/genbc.f->genBC()}. This process occurs in \path{compressible/b3res.f#28-163}. 

Applying \(\Sm\) on the LHS operates in a similar way, with the primary difference being the differences in how \(\Sm\) is applied analytically (ie. \(\Sm^T \G\) vs. \(\Sm^T \M \Sm\)). Since the \(\Sm^T \M \Sm\) operation acts on the applicable row and column of the \(\M\) matrix at once via \ttt{do} loops.

\section{Natural Boundary Conditions}

There are 6 different flux components: normal flux \(h^m\), pressure flux \(h^p\), viscous stress/traction vector \(h^v_j\), and heat flux \(h^h\).

\ttt{iBCB} is the equivalent of \ttt{iBC} for natural boundary conditions; it stores, bit-wise, which fluxes are to be set. \\
\ttt{BCB} stores the \(h^j\) values. The last index of the array represents the 6 flux components:

\begin{tabular} { |c|l|c|}
    \hline
    Index & Variable & Description \\
    \hline
    1 & \(h^m\) & mass/normal flux \\
    2 & \(h^p\) & pressure flux \\
    3 & \(h^v_1\) & viscous flux in x1 direction\\
    4 & \(h^v_2\) & viscous flux in x2 direction\\
    5 & \(h^v_3\) & viscous flux in x3 direction\\
    6 & \(h^h\) & heat flux \\
    \hline

\end{tabular}

\path{compressible/e3bvar.f#85-128} first computes the flow variable values at the quadrature points, \ttt{pres, T, u1, u2, u3, rho, ei, rk}. These values are used to set the ``floating'' fluxes (ie. fluxes on \(\Gamma-\Gamma_h^j\)). This also computes the user-defined flux values \(h^j\) at quadrature points and puts them into \ttt{rou, p, Fv1, Fv2, Fv3, heat} for the components of \ttt{BCB}. 

\path{compressible/e3b.f} sets a series of flux vectors \ttt{F1, F2, F3, F4, F5} which correspond to the 5 fluxes of the primitive equation (density flux, 3 velocity fluxes, and a temperature flux). These fluxes are either set to their ``floating'' values or the user specified flux values, as determined by the value of \ttt{iBCB}. The \ttt{e3b()} routine also defines \ttt{Fv2, Fv3, Fv4} and \ttt{Fv5} as parts of the viscous flux, which are then used for creating other fluxes (for example, part of the energy flux is simply \(u_i \tau_{ij}\)). 

Once the \ttt{F1, F2, F3, F4, F5} variables are set, they are then put into the residual \ttt{rl} in \path{compressible/e3b.f#266-292}.



\end{document}