
\documentclass[12pt, letterpaper, twoside]{article}
\usepackage[margin=2.50cm]{geometry} % Makes the Margins not stupid
\usepackage{bm} % for making bold equation variables
\usepackage{graphicx} % for placing images
\usepackage{float} % for making figure stay in place
\usepackage{amssymb} % for \lessapprox symbol
\usepackage{amsmath} % because using math and \text command in math mode
\usepackage[numbers]{natbib}
\usepackage{cancel} % for doing a diagonal slash cancel
\usepackage[mathscr]{euscript} % for Fancy curly F

\title{ASEN 5331 - HW3}
\author{James Wright}
% \date{2019-01-21}

%------------ MACROS ----------------
\newcommand{\etot}{e_{tot}}
\renewcommand{\vec}[1]{\bm{#1}}
\newcommand{\U}{\vec{U}}
\newcommand{\Y}{\vec{Y}}
\newcommand{\F}{\vec{F}}
\newcommand{\0}{\vec{0}}
\newcommand{\src}{\vec{\mathscr{F}}}

\begin{document}
\maketitle

Using the code that is posted on the web page, associate the following concepts to both your notes and to the code.  By associate, I mean a one- to- one correspondence 
between the variables in the notes and the variables in the code (complete with the range of their indices when they are arrays). Also, identify the routines where each 
of these variables get updated and thereby carry out the algorithms we have described in the notes (again mapping the algorithms from the notes to routines that carry them out)


\section{Gradient of shape function in the parent domain}

\section{Gradient of shape function in real domain}

\section{Metric tensor that maps 1) to 2).}

\section{Global Residual}

\section{Element Level Residual}

\section{Element Level tangent matrix}

\section{The gradient of the solution vector}

\section{Jacobian of the map from real domain to parent domain}

\section{The advective/convective flux}

\section{The Jacobian of the Advective/Convective Flux}
\end{document}