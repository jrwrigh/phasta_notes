\documentclass[11pt, letterpaper, twoside]{article}
\usepackage[margin=2.50cm]{geometry} % Makes the Margins not stupid
\usepackage{bm} % for making bold equation variables
\usepackage{graphicx} % for placing images
\usepackage{float} % for making figure stay in place
\usepackage{amssymb} % for \lessapprox symbol
\usepackage{amsmath} % because using math and \text command in math mode
\usepackage[numbers]{natbib}
\usepackage{cancel} % for doing a diagonal slash cancel
\usepackage[mathscr]{euscript} % for Fancy curly F
\usepackage{cleveref} % for \cref
\usepackage{commath} % for \od, \pd
\usepackage{outlines} % for outlines env
\usepackage{url} % for \path

\title{Notes on the Fundamental Equations of PHASTA}
\author{James Wright}
% \date{2019-01-21}

%------------ MACROS ----------------
\newcommand{\etot}{e_{tot}}
\renewcommand{\vec}[1]{\bm{#1}}
\newcommand{\A}{\vec{A}}
\newcommand{\U}{\vec{U}}
\newcommand{\G}{\vec{G}}
\newcommand{\Y}{\vec{Y}}
\newcommand{\F}{\vec{F}}
\newcommand{\W}{\vec{W}}
\newcommand{\x}{\vec{x}}
\newcommand{\NAx}{N_A(\x)}
\newcommand{\NBx}{N_B(\x)}
\newcommand{\NAix}{N_{A,i}(\x)}
\newcommand{\NBix}{N_{B,i}(\x)}
\newcommand{\Neax}{N^e_a(\x)}
\newcommand{\Nebx}{N^e_b(\x)}
\newcommand{\Neaix}{N^e_{a,i}(\x)}
\newcommand{\Nebix}{N^e_{b,i}(\x)}
\newcommand{\sumA}[1]{\sum_{A=1}^{#1}}
\newcommand{\sumB}[1]{\sum_{B=1}^{#1}}
\newcommand{\suma}[1]{\sum_{a=1}^{#1}}
\newcommand{\sumb}[1]{\sum_{b=1}^{#1}}
\newcommand{\src}{\vec{\mathscr{F}}}
\newcommand{\ttt}[1]{\texttt{#1}}
\DeclareMathOperator*{\assem}{\mathbb{A}}

\begin{document}
\maketitle

\section{Shorthand}
    \begin{equation*}
        \phi_{,t} = \frac{\partial \phi}{\partial t}
    \end{equation*}

    \begin{equation*}
        \phi_{,i} = \frac{\partial \phi}{\partial x_i}
    \end{equation*}

    \begin{equation*}
        u_{i,t} = \frac{\partial u_i}{\partial t}
    \end{equation*}

    \begin{equation*}
        u_{i,j} = \frac{\partial u_i}{\partial x_j}
    \end{equation*}

    \begin{equation*}
        \left[\phi u_i \right]_{,j} = \frac{\partial \phi u_i}{\partial x_j}
    \end{equation*}

\section{Theory}
\subsection{Fundamental Fluid Equations}
Unsteady Compressible Navier-Stokes (UCNS) equations:

    \subsubsection{Traditional (Strong) Form}
    \paragraph{Continuity}
        \begin{equation}\label{eq:fund_continuity}
            \rho_{,t} + \left[\rho u_j \right]_{,j} = 0
        \end{equation}

    \paragraph{Momentum}
        \begin{equation}\label{eq:fund_momentum}
            \left[\rho u_i \right]_{,t} + \left[\rho u_i u_j \right]_{,j} + p_{,i} = \tau_{ij,j} + b_i
        \end{equation}
        
    \paragraph{Energy}
        \begin{equation}\label{eq:fund_energy}
            \left[\rho \etot \right]_{,t} + 
            \left[\rho \etot u_j \right]_{,j} + \left[\rho u_j \right]_{,j} 
            = \left[\tau_{ij} u_j \right]_{,j} + b_i u_j + r + q_{i,i}
        \end{equation}
    
    \paragraph{Constituitive Equations} 
    \begin{equation}
        q_{i} = -\kappa T_{,i}
    \end{equation}
    
    \subsubsection{Conservative Vectorized Form}

    \begin{equation}
        \U \equiv
        \begin{Bmatrix}
            \rho \\
            \rho u_1 \\
            \rho u_2 \\
            \rho u_3 \\
            \rho \etot
        \end{Bmatrix}
        =
        \begin{Bmatrix}
            \rho \\
            \rho u_j \\
            \rho \etot
        \end{Bmatrix}
    \end{equation}

    Flux Vector:

    \begin{equation}
        \begin{split}
        \F_i & = 
        \underbrace{
        \begin{Bmatrix}
            \rho u_i \\
            \rho u_i u_j \\
            \rho u_i \etot
        \end{Bmatrix}
        + \begin{Bmatrix}
            0 \\
            p \delta_{ij}\\
            u_i p
        \end{Bmatrix} }_{\text{Advective Flux}}
        - \underbrace{ \begin{Bmatrix}
            0 \\
            \tau_{ij}\\
            \tau_{ij} u_j 
        \end{Bmatrix}
        + \begin{Bmatrix}
            0 \\
            \vec{0} \\
            q_{i,i}
        \end{Bmatrix} }_{\text{Diffusive Flux}} \\
        & = \F_i^{\text{adv}} + \F_i^{\text{dif}}
        \end{split}
    \end{equation}

    Also:
    \begin{equation}
        \F_i^{adv}  = 
        u_i \U
        + \begin{Bmatrix}
            0 \\
            p \delta_{ij}\\
            u_i p
        \end{Bmatrix} 
    \end{equation}

    Source Vector:

    \begin{equation}
        \src = 
        \begin{Bmatrix}
            0 \\
            b_j \\
            b_j u_j + r
        \end{Bmatrix}
    \end{equation}

    These terms combine together to form:
    \begin{equation} \label{eq:vecNS}
        \U_{,t} + \F_{i,i} = \src
    \end{equation}

\subsection{Finite Element Discretization}

    Before discretizing, we must first setup the UCNS equations. First, rearrange \cref{eq:vecNS} into a residual form:
    \begin{equation}
        \U_{,t} + \F_{i,i} - \src = \vec{0}
    \end{equation}

    Next multiply by weight/test functions, \(\W\). Note that \(\vec{0}\) is now just a scalar 0.
    \begin{equation}
        \W \cdot ( \U_{,t} + \F_{i,i} - \src )= 0
    \end{equation}

    To create the \textbf{Weak Form} of the NS equations, integrate over the domain \(\Omega\):
    \begin{equation}
        \int_\Omega \W \cdot (\U_{,t} + \F_{i,i} - \src) \dif \Omega= 0
    \end{equation}

    Next, perform integration by parts (IBP) and Gauss's theorem on \(\F_{i,i}\):
    \begin{equation} \label{eq:weakUCNS_IBP}
        \int_\Omega \left\{\W \cdot \U_{,t} - \W_{,i} \cdot \F_i(\U, \U_{,i}) - \W \cdot \src(\U) \right\} \dif \Omega 
        + \int_\Gamma \W \cdot \F_i(\U, \U_{,i}) \cdot \hat n_i \dif\Gamma= 0
    \end{equation}
    where \(\Gamma\) is the boundary of the domain \(\Omega\) and \(\hat n_i\) is the normal unit vector of the boundary surface. \Cref{eq:weakUCNS_IBP} represents the \textbf{Weak UCNS in IBP Form}. The fact that \(\F_i\) and \(\src\) are functions of \(\U\) and \(\U_{,i}\) is explicitly shown here.

    \subsubsection{Domain Discretization}
        Define nodes, points, and elements. 

        \paragraph{Shape Function Decomposition}
        Define a set of functions \(N\) that are a basis for the weight functions. In the case of the \textbf{Galerkin Form}, the basis/shape functions \(N\) are the same for the weight function and the solution function. We can decompose some constant \(\phi\) into

        \begin{equation}
            \phi(\x) = \sum_{A=1}^{n_n} N_A(\x) \phi_A
        \end{equation}

        where \(A\) is the index of each node, \(\phi_A\) is the value of \(\phi\) at each node \(A\), and \(n_n\) is the number of nodes used to discretize \(\Omega\). Note that \(\phi\) on the LHS is continuous, but \(\phi_A\) are discrete scalar values for every \(A\). This decomposition can be extrapolated to our unknowns: \(\U\), \(\U_{,t}\), and \(\W\):

        \begin{subequations}
            \label{eq:variableshapefuncdecomp}
           \begin{align}
                \U(\x) &= \sum_{A=1}^{n_n} N_A(\x) \U_A \\
                \U_{,t}(\x) &= \sum_{A=1}^{n_n} N_A(\x) \U_{A,t} \\
                \W(\x) &= \sum_{B=1}^{n_n} N_B(\x) \W_B 
           \end{align} 
        \end{subequations}

        Since we are working in Galerkin, \(N_A(\x) = N_B(\x) \) for \(A=B\) (ie, at the same node). 

        \paragraph{Discretize UCNS}
            Let's substitute the expressions in \cref{eq:variableshapefuncdecomp} into the weak UCNS form in \cref{eq:weakUCNS_IBP}.

            \begin{multline}
                \int_\Omega \Biggl\{ \sumB{n_n} \NBx \W_B \cdot \sumA{n_n} \NAx \U_{A,t} 
                - \sumB{n_n} \NBix \W_B \F_i \biggl(\sumA{n_n} \NAx \U_A,\ \sumA{n_n} \NAix \U_A\biggr) \\
                - \sumB{n_n} \W_B \src\biggl(\sumA{n_n} \NAx \U_A \biggr) \Biggr\} \dif \Omega \\
                + \int_\Gamma \sumB{n_n} \W_B \cdot \F_i\biggl(\sumA{n_n} \NAx \U_A,\ \sumA{n_n} \NAix \U_A\biggr) \hat n_i \dif \Gamma = 0
            \end{multline}

            Note that \(\U_{,i} = [\sumA{n_n} \NAx \U_A]_{,i}\) , but since \(\U_A\) is not a function of \(\x\) (it is only dependent on \(A\)), the sum and \(\U_A\) can be taken out of the brackets, leaving \(\NAx\) to be derived: \(\U_{,i} = \sumA{n_n} \NAix \U_A\).

            Since \(\W_B\) is a vector of constants, we can take it out of the integral, along with its respective summation:
            \begin{multline} \label{eq:shapefuncUCNS}
                \sumB{n_n} \W_B  \Biggl\{\int_\Omega \biggl\{  \NBx  \cdot \sumA{n_n} \NAx \U_{A,t} 
                -  \NBix  \F_i \biggl(\sumA{n_n} \NAx \U_A,\ \sumA{n_n} \NAix \U_A\biggr) \\
                -   \src\biggl(\sumA{n_n} \NAx \U_A \biggr) \biggr\} \dif \Omega \\
                + \int_\Gamma \F_i\biggl(\sumA{n_n} \NAx \U_A,\ \sumA{n_n} \NAix \U_A\biggr) \hat n_i \dif \Gamma \Biggr\} = 0
            \end{multline}

            \begin{equation}
                \cref{eq:shapefuncUCNS} \Rightarrow \sumB{n_n} \W_B \cdot \G_B = 0
            \end{equation}

            If we assume that \(\W_B\) is arbitrary, then \(\G_B=0\).

    \subsection{Local/Element level Formulation}
            For brevity, we will abbreviate the inputs of \(\F_i\) and \(\src\) to their continuous inputs: \(\F_i(\U, \U_i)\) and \(\src(\U)\).

            \begin{multline} 
                \G_b^e = \int_{\Omega^e} \Biggl\{ \Nebx \biggl[ \suma{n_{en}} \Neax \U_{a,t}^e - \src(\U) \biggr] - \Nebix \F_i(\U, \U_{,i}) \Biggr\} \dif \Omega^e \\
                + \int_{\Gamma^e} \Nebx \F_i(\U, \U_{,i}) \hat n_i \dif \Gamma^e
            \end{multline}

            Notes:
            \begin{outline}[enumerate]
                \1 \(n_{en}\) is the number of nodes per element
                \1 \(^e\) is the element ID number (\(e = \{1 ... n_n\}\)), 
                \1 \(_a\) is the node ID number relative to the current element, \(a = \{1 ... n_{en}\}\)
                \1 \(\U = \suma{n_{en}} \Neax \U_a^e\) , \(\U_i = \suma{n_{en}} \Neaix \U_a^e\)
                \1 \(\Gamma^e \subset \Gamma \). In words, \(\Gamma^e\) is a subset (small part) of the global \(\Gamma\) (ie. is not the boundary of \(\Omega^e\))
                
            \end{outline}

        \paragraph{Assembly (Local to Global)}
        The values at the local level must be brought up to the global level, in a process called assembly. This will be denoted by an \(\assem\). For a 1-D problem with 3 elements (and 4 nodes), this takes the form of:

        \begin{equation}
            \G_B = \assem \G_b^e \Rightarrow
            \begin{cases}
                \G_1 = \G_1^1 \\
                \G_2 = \G^1_2 + \G_1^2 \\
                \G_3 = \G_2^2 + \G^3_1 \\
                \G_4 = \G^3_2 \\
            \end{cases}
        \end{equation}

        Each \(\G_a^e\) in the array represents a single \(N_a\) shape function.



\section{Random Notes}

    RHS and LHS in the code refer to the Newton's method variation (see page 12b and 13 in notes). 

    RHS is the stabilized residual term:
    \begin{equation}
      \hat \G_B  
    \end{equation}
      and LHS is the mass matrix 
    \begin{equation}
      \sum \pd{\hat \G_B}{\Y}
    \end{equation}
    \subsection{Term Definition}
\begin{tabular} { |c|l|c|}
    \hline
    Term & Definition \\
    \hline
    \ttt{rlYi} & \(\A_i \Y_{,i}\)  \\
    \ttt{ri} & building of residual RHS (I think)  \\
    \ttt{EGmass} & mass matrix, \(\pd{\G}{\Y}\)  \\
    \hline

\end{tabular}

\subsection{Meaning of \ttt{n...}}

\begin{tabular} { |c|l|c|}
    \hline
    Term & Definition & Source/Relevant Reference \\
    \hline
    \ttt{nsd} & number of spacial dimensions & \path{common/common.h#343} \\
    \ttt{nflow} & number of flow variables (ie. size of \(\Y\)) & ? \\
    \ttt{nshape} & number of interior element shape functions & \path{common/common.h#444} \\
    \ttt{ngauss} & number of interior element integration points & \path{common/common.h#447} \\
    \ttt{npro} & number of elements processed in a single call of \ttt{e3.f} & Jansen lecture \\
    \ttt{npro} & number of virtual processors for the current block & \path{common/common/h#586} \\
    \ttt{nen} & maximum number of element nodes & \path{common/common.h#341} \\
    \ttt{nQpt} & number of quadrature points per element & \path{common/shp4t.f#14} \\
    \ttt{nshl} & number of shape functions per element & \path{common/genblkPosix.f#70} \\
    \ttt{nshg} & global number of shape functions & \path{common/common.h#354} \\
    \ttt{nenl} & number of element nodes for current block & \path{common/common.h#382} \\
    \ttt{nedof} & total number of degrees of freedom & \path{common/e3.f#35,344} \\

    \hline

\end{tabular}

    

\end{document}