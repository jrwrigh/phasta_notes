\documentclass[12pt, letterpaper, twoside]{article}
\usepackage[margin=2.50cm]{geometry} % Makes the Margins not stupid
\usepackage{bm} % for making bold equation variables
\usepackage{graphicx} % for placing images
\usepackage{float} % for making figure stay in place
\usepackage{amssymb} % for \lessapprox symbol
\usepackage{amsmath} % because using math and \text command in math mode
\usepackage[numbers]{natbib}
\usepackage{cancel} % for doing a diagonal slash cancel
\usepackage[mathscr]{euscript} % for Fancy curly F

\title{Notes on the Fundamental Equations of PHASTA}
\author{James Wright}
% \date{2019-01-21}

%------------ MACROS ----------------
\newcommand{\etot}{e_{tot}}
\renewcommand{\vec}[1]{\bm{#1}}
\newcommand{\U}{\vec{U}}
\newcommand{\Y}{\vec{Y}}
\newcommand{\F}{\vec{F}}
\newcommand{\0}{\vec{0}}
\newcommand{\src}{\vec{\mathscr{F}}}

\begin{document}
\maketitle

\section{Shorthand}
    \begin{equation}
        \phi_{,t} = \frac{\partial \phi}{\partial t}
    \end{equation}

    \begin{equation}
        \phi_{,i} = \frac{\partial \phi}{\partial x_i}
    \end{equation}

    \begin{equation}
        u_{i,t} = \frac{\partial u_i}{\partial t}
    \end{equation}

    \begin{equation}
        u_{i,j} = \frac{\partial u_i}{\partial x_j}
    \end{equation}

    \begin{equation}
        \left[\phi u_i \right]_{,j} = \frac{\partial \phi u_i}{\partial x_j}
    \end{equation}

\section{Fundamental Fluid Equations}
Compressible Navier-Stokes equations:

    \subsection{Traditional Form}
    \paragraph{Continuity}
        \begin{equation}\label{eq:fund_continuity}
            \rho_{,t} + \left[\rho u_j \right]_{,j} = 0
        \end{equation}

    \paragraph{Momentum}
        \begin{equation}\label{eq:fund_momentum}
            \left[\rho u_i \right]_{,t} + \left[\rho u_i u_j \right]_{,j} + p_{,i} = \tau_{ij,j} + b_i
        \end{equation}
        
    \paragraph{Energy}
        \begin{equation}\label{eq:fund_energy}
            \left[\rho \etot \right]_{,t} + 
            \left[\rho \etot u_j \right]_{,j} + \left[\rho u_j \right]_{,j} 
            = \left[\tau_{ij} u_j \right]_{,j} + b_i u_j + r + q_{i,i}
        \end{equation}
    
    \subsection{Conservative Vectorized Form}

    \begin{equation}
        \U \equiv
        \begin{Bmatrix}
            \rho \\
            \rho u_1 \\
            \rho u_2 \\
            \rho u_3 \\
            \rho \etot
        \end{Bmatrix}
        =
        \begin{Bmatrix}
            \rho \\
            \rho u_j \\
            \rho \etot
        \end{Bmatrix}
    \end{equation}

    Flux Vector:

    \begin{equation}
        \begin{split}
        \F_i & = 
        \underbrace{
        \begin{Bmatrix}
            \rho u_i \\
            \rho u_i u_j \\
            \rho u_i \etot
        \end{Bmatrix}
        + \begin{Bmatrix}
            0 \\
            p \delta_{ij}\\
            u_i p
        \end{Bmatrix} }_{\text{Advective Flux}}
        - \underbrace{ \begin{Bmatrix}
            0 \\
            \tau_{ij}\\
            \tau_{ij} u_j 
        \end{Bmatrix}
        + \begin{Bmatrix}
            0 \\
            \0 \\
            \tau_{ij} u_j 
        \end{Bmatrix} }_{\text{Diffusive Flux}} \\
        & = \F_i^{\text{adv}} + \F_i^{\text{dif}}
        \end{split}
    \end{equation}

    Also:
    \begin{equation}
        \F_i^{adv}  = 
        u_i \U
        + \begin{Bmatrix}
            0 \\
            p \delta_{ij}\\
            u_i p
        \end{Bmatrix} 
    \end{equation}

    Source Vector:

    \begin{equation}
        \src = 
        \begin{Bmatrix}
            0 \\
            b_j \\
            b_j u_j + r
        \end{Bmatrix}
    \end{equation}

    These terms combine together to form:
    \begin{equation}
        \U_{,t} + \F_{i,i} = \src
    \end{equation}

\end{document}