\documentclass{ucb}

\begin{document}

\ucbcourse{\textbf{ASEN 5331} CFD: Unstructured Grid}
\ucbtitle{Homework 2}
% \ucbgroup{X}
\ucbdate{01}{10}{2019}
\ucbauthors{
    Pol Mesalles Ripoll \\
}
\ucblocation{Boulder, Colorado}
\ucbcover

\section*{Questions}

\begin{enumerate}

\item \textit{Express the diffusive flux in terms of the primitive variables $\bm{Y} = \begin{Bmatrix}p & \bm{u} & T\end{Bmatrix}^T$ and the gradient of the primitive variables.}
\label{item1}

Let's start by presenting the expression for the diffusive flux in the $i$-th direction:
\begin{equation}
    \bm{F}_i^\mathrm{diff} = 
    \begin{Bmatrix}
        F_{i1} \\
        F_{i(m+1)} \\
        F_{i5}
    \end{Bmatrix}
    =
    -
    \begin{Bmatrix}
        0 \\
        \tau_{im} \\
        u_n\tau_{in}
    \end{Bmatrix}
    +
    \begin{Bmatrix}
        0 \\
        0_m \\
        q_i
    \end{Bmatrix}
    \label{eq:diff}
\end{equation}
where $i = 1 \ldots 3$ counts the flux directions and $m = 1 \ldots 3$ counts each component of the three momentum equations.

Our vector of primitive variables is:
\begin{equation}
    \bm{Y} =
    \begin{Bmatrix}
        Y_1 \\
        Y_{k+1} \\
        Y_5
    \end{Bmatrix}
    =
    \begin{Bmatrix}
        p \\
        u_k \\
        T
    \end{Bmatrix}
\end{equation}
where $k = 1 \ldots 3$ counts each component of the velocity vector.

We will make use of the following constitutive relations:
\begin{subequations}
    \begin{gather}
        \tau_{im} = \mu S_{im} = \mu\left(u_{i,m} + u_{m,i}\right) + \lambda u_{l,l}\delta_{im} \\
        q_i = -\kappa T_{,i}
    \end{gather}
    \label{eq:const}
\end{subequations}
where the sign in Fourier'ls law has been deliberately omitted given the convention used in class.

Putting it all together, we can write the diffusive flux in terms of the variables in $\bm{Y}$ as follows:
\begin{equation}
    \bm{F}_i^\mathrm{diff} = 
    \begin{Bmatrix}
        F_{i1} \\
        F_{i(m+1)} \\
        F_{i5}
    \end{Bmatrix}
    =
    -
    \begin{Bmatrix}
        0 \\
        \mu\left(Y_{i+1,m} + Y_{m+1,i}\right) + \lambda Y_{l+1,l}\delta_{im} \\
        Y_{n+1} \left[\mu\left(Y_{i+1,n} + Y_{n+1,i}\right) + \lambda Y_{l+1,l}\delta_{in}\right]
    \end{Bmatrix}
    +
    \begin{Bmatrix}
        0 \\
        0_m \\
        -\kappa Y_{5,i}
    \end{Bmatrix}
    \label{eq:diffY}
\end{equation}

\pagebreak

\item \textit{Find the entries in the diffusive tangent matrix $\bm{K}_{ij}$, as described in class.}
\label{item2}

The diffusive tangent matrix is defined as the tensor $\bm{K}_{ij}$ such that:
\begin{equation}
    \bm{F}_i^\mathrm{diff} = -\bm{K}_{ij} \bm{Y}_{,j}
\end{equation}

By factoring out the primitive variables from \autoref{eq:diffY}, one can write:
\begin{equation}
    \underbrace{
        \begin{Bmatrix}
            F_{i1} \\
            F_{i(m+1)} \\
            F_{i5}
        \end{Bmatrix}
    }_{\bm{F}_i^\mathrm{diff}}
    =
    -
    \underbrace{
        \begin{bmatrix}
            0 & 0 & 0 \\
            0 & K_{ij(m+1)(k+1)} & 0 \\
            0 & K_{ij5(k+1)} & K_{ij55}
        \end{bmatrix}
    }_{\bm{K}_{ij}}
    \underbrace{
        \begin{Bmatrix}
            Y_{1,j} \\
            Y_{k+1,j} \\
            Y_{5,j}
        \end{Bmatrix}
    }_{\bm{Y}_{,j}}
    \label{eq:Kij}
\end{equation}
where:
\begin{align}
    &\begin{aligned}
        K_{ij(m+1)(k+1)} &= \mu\left(\delta_{(i+1)(k+1)}\delta_{mj} + \delta_{(m+1)(k+1)}\delta_{ij}\right) + \lambda\,\delta_{(l+1)(k+1)}\delta_{lj}\delta_{im} \\
        &= \mu\left(\delta_{ik}\delta_{mj} + \delta_{mk}\delta_{ij}\right) + \lambda\,\delta_{kj}\delta_{im}
    \end{aligned}
    \\
    &\begin{aligned}
        K_{ij5(k+1)} &= Y_{n+1}\left[\mu\left(\delta_{(i+1)(k+1)}\delta_{nj} + \delta_{(n+1)(k+1)}\delta_{ij}\right) + \lambda\,\delta_{(l+1)(k+1)}\delta_{lj}\delta_{in}\right] \\
        &= Y_{n+1}\left[\mu\left(\delta_{ik}\delta_{nj} + \delta_{nk}\delta_{ij}\right) + \lambda\,\delta_{kj}\delta_{in}\right]
    \end{aligned}
    \\
    &K_{ij55} = \kappa \, \delta_{ij}
\end{align}
and where the following free indices have been used:
\begin{itemize}
    \item $i = 1 \ldots 3$ counts each flux direction
    \item $j = 1 \ldots 3$ counts each gradient direction
    \item $m = 1 \ldots 3$ counts each row of the $\bm{K}_{ij}$ matrix (and each row of the $\bm{F}_i^\mathrm{diff}$ vector)
    \item $k = 1 \ldots 3$ counts each column of the $\bm{K}_{ij}$ matrix (and each row of the $\bm{Y}_{,j}$ or $\bm{Y}$ vectors)
\end{itemize}
along with $l = 1 \ldots 3$ and $n = 1 \ldots 3$ that are used as "dummy" indices.

\pagebreak

\item \textit{Express the diffusive flux in terms of the conservation variables and the gradient of the conservation variables. Discuss the expected complexity of repeating step \ref{item2} for this variable choice. Do a couple of terms to make your point clear.}
\label{item3}

As a first step, recall the vector of conservation variables:
\begin{equation}
    \bm{U} =
     \begin{Bmatrix}
        U_1 \\
        U_{k+1} \\
        U_5
    \end{Bmatrix}
    =
    \begin{Bmatrix}
        \rho \\
        \rho u_k \\
        \rho e_\mathrm{tot}
    \end{Bmatrix}
\end{equation}

In this case, besides the relations from \autoref{eq:const}, we will make use of:
\begin{subequations}
    \begin{gather}
        e_\mathrm{tot} = e + \frac{1}{2}u_pu_p \\
        e = c_v T
    \end{gather}
\end{subequations}

Then, the resulting expression for the diffusive flux in terms of the variables in $\bm{U}$ is the following:
\begin{equation}
    \begin{aligned}
        \bm{F}_i^\mathrm{diff} &= 
        \begin{Bmatrix}
            F_{i1} \\
            F_{i(m+1)} \\
            F_{i5}
        \end{Bmatrix}
        = \\
        &-
        \begin{Bmatrix}
            0 \\
            \mu\left[\left(\frac{U_{i+1,m}}{U_1} - \frac{U_{i+1}U_{1,m}}{U_1^2}\right) + \left(\frac{U_{m+1,i}}{U_1} - \frac{U_{m+1}U_{1,i}}{U_1^2}\right)\right] + \lambda\left(\frac{U_{l+1,l}}{U_1} - \frac{U_{l+1}U_{1,l}}{U_1^2}\right)\delta_{im} \\
            \frac{U_{n+1}}{U_1} \left(\mu\left[\left(\frac{U_{i+1,n}}{U_1} - \frac{U_{i+1}U_{1,n}}{U_1^2}\right) + \left(\frac{U_{n+1,i}}{U_1} - \frac{U_{n+1}U_{1,i}}{U_1^2}\right)\right] + \lambda\left(\frac{U_{l+1,l}}{U_1} - \frac{U_{l+1}U_{1,l}}{U_1^2}\right)\delta_{in}\right)
        \end{Bmatrix} \\
        &-
        \begin{Bmatrix}
            0 \\
            0_m \\
            \frac{\kappa}{c_v}\left[\left(\frac{U_{5,i}}{U_1} - \frac{U_5U_{1,i}}{U_1^2}\right) - \left(\frac{U_{p+1,i}U_{p+1}}{U_1^2} - \frac{U_{p+1}U_{p+1}U_{1,i}}{U_1^3}\right)\right]
        \end{Bmatrix}
    \end{aligned}
    \label{eq:diffU}
\end{equation}

As can be seen from the resulting expression \eqref{eq:diffU}, with this choice of $\bm{U}$ (conservation) variables, the diffusive tangent matrix $\bm{K}_{ij}$ will not factor out easily.

Next, write the new system of equations equivalent to \autoref{eq:Kij}:
\begin{equation}
    \underbrace{
        \begin{Bmatrix}
            F_{i1} \\
            F_{i(m+1)} \\
            F_{i5}
        \end{Bmatrix}
    }_{\bm{F}_i^\mathrm{diff}}
    =
    -
    \underbrace{
        \begin{bmatrix}
            0 & 0 & 0 \\
            K_{ij(m+1)1} & K_{ij(m+1)(k+1)} & 0 \\
            K_{ij51} & K_{ij5(k+1)} & K_{ij55}
        \end{bmatrix}
    }_{\bm{K}_{ij}}
    \underbrace{
        \begin{Bmatrix}
            U_{1,j} \\
            U_{k+1,j} \\
            U_{5,j}
        \end{Bmatrix}
    }_{\bm{U}_{,j}}
\end{equation}
where we can see there are already more non-zero terms than before. A couple of these are:
\begin{equation}
    \begin{aligned}
        &K_{ij(m+1)1} = -\left(\frac{U_{i+1}}{U_1^2}\,\delta_{mj} + \frac{U_{m+1}}{U_1^2}\,\delta_{ij}\right) -\lambda\,\frac{U_{l+1}}{U_1^2}\,\delta_{lj}\delta_{im} \\
        &K_{ij55} = \frac{\kappa}{c_v}\frac{1}{U_1}\,\delta_{ij}
    \end{aligned}
\end{equation}
and one can see that the momentum terms will be quite long. The indices used here are the same from Question \ref{item2}, adding $p = 1 \ldots 3$ to the list of "dummy indices" (for the kinetic energy velocity components).

\end{enumerate}

\end{document}
